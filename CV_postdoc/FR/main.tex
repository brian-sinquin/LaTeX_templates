%%%%%%%%%%%%%%%%%
% This is an sample CV template created using altacv.cls
% (v1.1.5, 1 December 2018) written by LianTze Lim (liantze@gmail.com). Now compiles with pdfLaTeX, XeLaTeX and LuaLaTeX.
%
%% It may be distributed and/or modified under the
%% conditions of the LaTeX Project Public License, either version 1.3
%% of this license or (at your option) any later version.
%% The latest version of this license is in
%%    http://www.latex-project.org/lppl.txt
%% and version 1.3 or later is part of all distributions of LaTeX
%% version 2003/12/01 or later.
%%%%%%%%%%%%%%%%

%% If you need to pass whatever options to xcolor
\PassOptionsToPackage{dvipsnames}{xcolor}

%% If you are using \orcid or academicons
%% icons, make sure you have the academicons
%% option here, and compile with XeLaTeX
%% or LuaLaTeX.
% \documentclass[10pt,a4paper,academicons]{altacv}

%% Use the "normalphoto" option if you want a normal photo instead of cropped to a circle
% \documentclass[10pt,a4paper,normalphoto]{altacv}

\documentclass[9pt,a4paper,academicons]{altacv}


%\usepackage[bibencoding=utf8]{biblatex} %biber war fehlerhaft / sorting wird verwendet damit es der Reihenfolge entpsricht
\addbibresource{main}

%% AltaCV uses the fontawesome and academicon fonts
%% and packages.
%% See texdoc.net/pkg/fontawecome and http://texdoc.net/pkg/academicons for full list of symbols. You MUST compile with XeLaTeX or LuaLaTeX if you want to use academicons.

% Change the page layout if you need to
\geometry{left=1.3cm,right=8.5cm,marginparwidth=6cm,marginparsep=1.2cm,top=1.25cm,bottom=1.25cm}

% Change the font if you want to, depending on whether
% you're using pdflatex or xelatex/lualatex
\ifxetexorluatex
  % If using xelatex or lualatex:
  \setmainfont{Carlito}
\else
  % If using pdflatex:
\usepackage{hyperref}
  \usepackage[utf8]{inputenc}
  \usepackage[T1]{fontenc}
  \usepackage[default]{lato}
\fi

% Change the colours if you want to

% \definecolor{dark}{HTML}{0f6856}
% \definecolor{ldark}{HTML}{32918d}
% \definecolor{SlateGrey}{HTML}{0f3936}
% \definecolor{LightGrey}{HTML}{444444}

\definecolor{dark}{HTML}{222255}
\definecolor{ldark}{HTML}{222255}
\definecolor{SlateGrey}{HTML}{222266}
\definecolor{LightGrey}{HTML}{444455}

\usepackage{orcidlink}


\colorlet{heading}{SlateGrey}
\colorlet{accent}{SlateGrey}
\colorlet{emphasis}{dark}
\colorlet{body}{black}

% Change the bullets for itemize and rating marker
% for \cvskill if you want to
\renewcommand{\itemmarker}{{\small\textbullet}}
\renewcommand{\ratingmarker}{\faCircle}

\usepackage{hyperref}
\hypersetup{
	colorlinks=true,
	linkcolor=SlateGrey,
	filecolor=magenta,      
	urlcolor=ldark,
}


\begin{document}
\name{\headingfont{Brian Sinquin}}
\tagline{Doctorant en physique des LASERs / Ph.D Student in Photonics\\
  \small{INSA, Institut FOTON (Université Rennes 1) UMR 6082 \'Equipe DOP}\\
  \vspace{1em}
  \color{SlateGrey}{``\small{\textit{ Oscillateur optoélectronique à modulation directe. Génération de signaux micro-ondes et d'impulsions optiques courtes }}}''
}
\photo{2.8cm}{ID}
\personalinfo{%
  % Not all of these are required!
  % You can add your own with \printinfo{symbol}{detail}
  \email{brian.sinquin@gmail.com}
  \phone{06 52 40 41 62}
  \location{3 rue Robert Rême, 35000 Rennes}
  \voiture{Permis B}
  {{\Large\orcidlink{0000-0001-8278-394X}}\hspace{0.5em}0000-0001-8278-394X \hspace{2em}}
  \linkedin{\href{https://linkedin.com/in/brian-sinquin/}{linkedin.com/in/brian-sinquin/}}
  % \github{\href{https://github.com/ParadiseLab/}{ParadiseLab}}

  % \github{github.com/brian.sinquin}
  %% You MUST add the academicons option to \documentclass, then compile with LuaLaTeX or XeLaTeX, if you want to use \orcid or other academicons commands.
  % \orcid{orcid.org/0000-0000-0000-0000}
}

%% Make the header extend all the way to the right, if you want.
\begin{fullwidth}
  \makecvheader
\end{fullwidth}

%% Depending on your tastes, you may want to make fonts of itemize environments slightly smaller
% \AtBeginEnvironment{itemize}{\small}

%% Provide the file name containing the sidebar contents as an optional parameter to \cvsection.
%% You can always just use \marginpar{...} if you do
%% not need to align the top of the contents to any
%% \cvsection title in the "main" bar.

\cvsection[page1sidebar]{Expériences}


\cvevent{Doctorat (soutenance prévue le 14/12/2023)}{INSA,  Univ. Rennes, Institut FOTON UMR 6082 \'Equipe DOP}{2020 -- 2023}{Rennes}
\textbf{Oscillateur optoélectronique à modulation directe. Génération de signaux micro-ondes et d'impulsions optiques courtes}\smallskip

Travaux dirigés par \textbf{Dr. Marco Romanelli} (marco.romanelli@univ-rennes.fr)
\small{
	\begin{itemize}
		\item Simulation numérique
		\item Caractérisation photonique
		\item Développement logiciel et traitement du signal
	\end{itemize}
}

\divider

\cvevent{Stage de Master 2}{SensUp (by Lumibird)}{février -- juillet 2020}{Lannion}

\textbf{Optimisation et caractérisation d'une chaîne LiDAR hétérodyne}
\small{
	\begin{itemize}
		\item Simulation numérique
		\item Caractérisation photonique
		\item Développement logiciel et traitement du signal
	\end{itemize}
}

\divider

\cvevent{Stage de Master 1}{OPTIMAG (UBO)}{avril -- juin 2019}{Brest}

\textbf{Mesure ultra-rapide du pouvoir rotatoire par codage spectral de la
	polarisation}
\small{
	\begin{itemize}
		\item Simulation numérique
		\item Polarisation
		\item Influence du bruit
	\end{itemize}
}




\cvsection{Formations}

\cvevent{Master 2 Parcours Photonique}{Université de Bretagne Occidentale}{2019 -- 2020}{Brest}
\textit{Mention bien (Major)}\smallskip
\small{
  \begin{itemize}
    \item Optique intégrée
    \item LASERs et Télécommunication
    \item Propagation optique \& Milieux diffusants
    \item Projet bibliographique : Génération de super-continuums dans les fibres micro-structurées
  \end{itemize}
}

\divider

\cvevent{Master 1 Physique fondamentale et appliquée}{Université de Bretagne Occidentale}{2018 -- 2019}{Brest}
\textit{Mention assez bien (Major)}\smallskip
\small{
  \begin{itemize}
    \item Théorie du signal
    \item Optique non linéaire, anisotrope
    \item Physique statistique
    \item Matière condensée
  \end{itemize}
}
\divider

\cvevent{Licence de physique}{Université de Bretagne Occidentale}{2015 -- 2018}{Brest}
\textit{Mention très bien en L3 (Major)}\smallskip
\small{
  \begin{itemize}
    \item Optique ondulatoire
    \item Mécanique quantique
    \item Ondes et matière
    \item Physique expérimentale et numérique
    \item Projet expérimental de L3 en laboratoire : Caractérisation de biomatériaux par colorimétrie
  \end{itemize}
}





%%\divider

%%\cvevent{Agent d'accueil}{CROUS}{septembre 2018 -- juin 2019}{Brest}
%%\textbf{Centre Régional des Œuvres Universitaires et Scolaires}
%%\small{
%%\begin{itemize}
%%\item Accueil
%%\item Distribution du courrier
%%\item Constitution de dossiers administratifs
%%\item Paiements bancaires (CB, chèques)
%%\end{itemize}
%%}

\medskip


\newpage

\begin{fullwidth}

  \cvsection{Publications dans des revues spécialisées}



  \begin{itemize}
    \item \normalsize \fullcite{Sinquin2021}
    \item \normalsize \fullcite{Sinquin2023}
  \end{itemize}

  % \nocite{*}

  % \begingroup
  % \renewcommand{\section}[2]{}%
  % %\renewcommand{\chapter}[2]{}% for other classes
  % \printbibliography
  % \endgroup
  \cvsection{Congrès internationaux à comités de lecture}
  \cvevent{CLEO\textregistered/Europe-EQEC 2021}{Conference on Lasers and Electro-Optics}{21--25 juin 2021}{}

\begin{itemize} \item \fullcite{cleo2021}   \end{itemize}
  
\smallskip
  
  \href{https://ieeexplore.ieee.org/document/9542636}{Présentation orale de 15 minutes en visio-conférence (COVID)}
  
  \divider
  
  
  
  
  \cvevent{CLEO\textregistered/Europe-EQEC 2023}{Conference on Lasers and Electro-Optics}{26--30 juin 2023}{Munich -- Allemagne}
  
\begin{itemize} \item \fullcite{cleo2023a}   \end{itemize}

\smallskip
  
  
  \href{https://doi.org/10.1109/CLEO/Europe-EQEC57999.2023.10231990}{Présentation orale de 15 minutes}
  
  \divider
  
  \cvevent{CLEO\textregistered/Europe-EQEC 2023}{Conference on Lasers and Electro-Optics}{26--30 juin 2023}{Munich -- Allemagne}
  
\begin{itemize} \item \fullcite{cleo2023b}   \end{itemize}

\smallskip
  
  \href{https://doi.org/10.1109/CLEO/Europe-EQEC57999.2023.10231566}{Présentation orale de 15 minutes}
  
  \cvsection{Congrès nationaux à comités de lecture}
  
  \cvevent{Journée du Club Optique Micro-ondes 2021}{Société Française d'Optique}{4 juin 2021}{}
{\quote{Signaux Opto-RF très bas bruit de phase et instabilités dynamiques d’un OEO à modulation directe}}

\smallskip

\href{https://hal.archives-ouvertes.fr/FOTON-DOP/hal-03285993}{Présentation orale de 15 minutes en visio-conférence (COVID)}

\divider


\cvevent{OPTIQUE Dijon 2021}{Société Française d'Optique}{5--9 juillet 2021}{Dijon -- France}
{\quote{Oscillateur Opto-Électronique à modulation directe de faible bruit de phase}}

\smallskip

\href{https://hal.archives-ouvertes.fr/FOTON-DOP/hal-03284744}{Présentation de poster lors de l'événement}

\divider

\cvevent{OPTIQUE Nice 2022}{Société Française d'Optique}{4--8 juillet 2022}{Nice -- France}

{\quote{Génération de peignes de fréquence et d’impulsions dans un Oscillateur Opto-Électronique à modulation directe}}

\smallskip

\href{https://hal.science/hal-03988116}{Présentation orale de 15 minutes}

\divider



\cvevent{Journée du Club Optique Micro-ondes 2022}{Société Française d'Optique}{13 juin 2022}{Besançon -- France}
{\quote{Oscillateur optoélectronique (OEO) générant des peignes de fréquences et des trains d’impulsions optiques}}
\medskip

\href{https://hal.science/hal-03986413}{Présentation de poster lors de l'évenement}

\divider

\cvevent{Journée du Club Optique Micro-ondes 2023}{Société Française d'Optique}{19 juin 2023}{}
{\quote{Direct-Modulation OEO for Optical Pulses and Frequency combs generation}}

\medskip

\href{https://hal.science/hal-
	04133619}{Présentation de poster lors de l'évenement en visio-conférence}

\newpage

  \cvsection{Autres communications scientifiques}

    \cvevent{Antennes et circuits: des micro-ondes aux ondes millimétriques et THz}{GDR Ondes 2021}{18 mars 2021}{}
    
{\quote{Low phase noise direct-modulation Optoelectronic Oscillator}}

\smallskip

\href{https://www.youtube.com/watch?v=NiCSm7F7ba8}{Présentation orale de 15 minutes en visio-conférence (COVID)}





 

  \cvsection{Enseignement}

  \cvevent{Enseignements en première année de thèse}{Université de Rennes 1}{2020--2021}{Rennes -- France}

  \begin{itemize}
    \item TP Optique géométrique \textbf{(Licence 1)} -- 16h
    \item TD Systèmes et Signaux pour la Physique \textbf{(Licence 3)} -- 10h

  \end{itemize}

  \divider

  \cvevent{Enseignements en deuxième année de thèse}{Université de Rennes 1}{2021--2022}{Rennes -- France}

  \begin{itemize}
    \item TP LASER \textbf{(Master 1)} -- 4h
    \item TD \'Eléctromagnétisme \textbf{(Licence 2)} -- 20h
    \item TD Systèmes et Signaux pour la Physique \textbf{(Licence 3)} -- 10h

  \end{itemize}
  \divider
  
    \cvsection{Encadrement}
\end{fullwidth}
%% If the NEXT page doesn't start with a \cvsection but you'd
%% still like to add a sidebar, then use this command on THIS
%% page to add it. The optional argument lets you pull up the
%% sidebar a bit so that it looks aligned with the top of the
%% main column.
% \addnextpagesidebar[-1ex]{page3sidebar}


\end{document}
