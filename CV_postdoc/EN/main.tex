%%%%%%%%%%%%%%%%%
% This is an sample CV template created using altacv.cls
% (v1.1.5, 1 December 2018) written by LianTze Lim (liantze@gmail.com). Now compiles with pdfLaTeX, XeLaTeX and LuaLaTeX.
%
%% It may be distributed and/or modified under the
%% conditions of the LaTeX Project Public License, either version 1.3
%% of this license or (at your option) any later version.
%% The latest version of this license is in
%%    http://www.latex-project.org/lppl.txt
%% and version 1.3 or later is part of all distributions of LaTeX
%% version 2003/12/01 or later.
%%%%%%%%%%%%%%%%

%% If you need to pass whatever options to xcolor
\PassOptionsToPackage{dvipsnames}{xcolor}

%% If you are using \orcid or academicons
%% icons, make sure you have the academicons
%% option here, and compile with XeLaTeX
%% or LuaLaTeX.
% \documentclass[10pt,a4paper,academicons]{altacv}

%% Use the "normalphoto" option if you want a normal photo instead of cropped to a circle
% \documentclass[10pt,a4paper,normalphoto]{altacv}

\documentclass[9pt,a4paper,academicons]{altacv}


%\usepackage[bibencoding=utf8]{biblatex} %biber war fehlerhaft / sorting wird verwendet damit es der Reihenfolge entpsricht
\addbibresource{main}

%% AltaCV uses the fontawesome and academicon fonts
%% and packages.
%% See texdoc.net/pkg/fontawecome and http://texdoc.net/pkg/academicons for full list of symbols. You MUST compile with XeLaTeX or LuaLaTeX if you want to use academicons.

% Change the page layout if you need to
\geometry{left=1.3cm,right=8.5cm,marginparwidth=6cm,marginparsep=1.2cm,top=1.25cm,bottom=1.25cm}

% Change the font if you want to, depending on whether
% you're using pdflatex or xelatex/lualatex
\ifxetexorluatex
  % If using xelatex or lualatex:
  \setmainfont{Carlito}
\else
  % If using pdflatex:
\usepackage{hyperref}
  \usepackage[utf8]{inputenc}
  \usepackage[T1]{fontenc}
  \usepackage[default]{lato}
\fi

% Change the colours if you want to

% \definecolor{dark}{HTML}{0f6856}
% \definecolor{ldark}{HTML}{32918d}
% \definecolor{SlateGrey}{HTML}{0f3936}
% \definecolor{LightGrey}{HTML}{444444}

\definecolor{dark}{HTML}{222288}
\definecolor{ldark}{HTML}{222255}
\definecolor{SlateGrey}{HTML}{222277}


\usepackage{orcidlink}


\colorlet{heading}{SlateGrey}
\colorlet{accent}{SlateGrey}
\colorlet{emphasis}{dark}
\colorlet{body}{black}

% Change the bullets for itemize and rating marker
% for \cvskill if you want to
\renewcommand{\itemmarker}{{\small\textbullet}}
\renewcommand{\ratingmarker}{\faCircle}

\usepackage{hyperref}
\hypersetup{
	colorlinks=true,
	linkcolor=dark,
	filecolor=magenta,      
	urlcolor=dark,
}




\begin{document}
\name{{\headingfont{Brian Sinquin}}  \Huge / {\headingfont{Ph.D in photonics}}}
\tagline{\textit{Looking for a postdoctoral position}}
%\tagline{
%	Doctorant en physique des LASERs / Ph.D Student in Photonics\\
%  \small{INSA, Institut FOTON (Université Rennes 1) UMR 6082 \'Equipe DOP}\\
%  \vspace{1em}
%  \color{SlateGrey}{``\small{\textit{ Oscillateur optoélectronique à modulation directe. Génération de signaux micro-ondes et d'impulsions optiques courtes }}}''
%}
\photo{2.8cm}{ID}
\personalinfo{%
  % Not all of these are required!
  % You can add your own with \printinfo{symbol}{detail}
  \large{\twemoji[width=1em]{fr}\hspace{0.9em}French\hspace{1.5em}Age: 27}\hspace{1.5em}
  \phone{(+33)6 52 40 41 62}
  \location{Rennes, France} % 3 rue Robert Rême, 
  %\voiture{Permis B}
  \email{brian.sinquin@gmail.com}
  %\orcid{0000-0001-8278-394X}
  \linkedin{\href{https://linkedin.com/in/brian-sinquin/}{linkedin.com/in/brian-sinquin/}}
  % \github{\href{https://github.com/ParadiseLab/}{ParadiseLab}}

  % \github{github.com/brian.sinquin}
  %% You MUST add the academicons option to \documentclass, then compile with LuaLaTeX or XeLaTeX, if you want to use \orcid or other academicons commands.
  % \orcid{orcid.org/0000-0000-0000-0000}
}

%% Make the header extend all the way to the right, if you want.
\begin{fullwidth}
  \makecvheader
\end{fullwidth}

%% Depending on your tastes, you may want to make fonts of itemize environments slightly smaller
% \AtBeginEnvironment{itemize}{\small}

%% Provide the file name containing the sidebar contents as an optional parameter to \cvsection.
%% You can always just use \marginpar{...} if you do
%% not need to align the top of the contents to any
%% \cvsection title in the "main" bar.

\cvsection[page1sidebar]{Work Experience}


\cvevent{Ph.D (defended the 14/12/2023)}{INSA,  Univ. Rennes, Institut FOTON UMR 6082 \'Equipe DOP}{2020 -- 2023}{Rennes, France}
\textbf{Direct-modulation optoelectronic oscillator. Microwave synthesis and short optical pulses generation} | \href{https://hal.science/tel-04461814v1/}{tel-04461814v1}\smallskip

Supervisor: \textbf{Dr. Marco Romanelli} (marco.romanelli@univ-rennes.fr)
\smallskip
\small{
  \begin{itemize}
    \item Realisation and optimisation of a direct-modulation \textbf{OEO} (OptoElectronic Oscillator) showing a very low phase noise 10 GHz microwave siqnal (single and dual loop architectures).

    \item Design and realization of optical pulse compression in optical fibers (dispersion, Kerr non-linearity), generation of very low jitter picosecond pulses at a rate of 10 GHz.
    \item Photonics (OSA, Auto-correlateur, RIN) and microwave (ESA, VNA, phase-noise, amplifiers, RF filters) characterizations in collaboration with the technical support team (engineers in optics, electronics).
    \item Novel, accurate and simple method for the measurement of the linewidth enhancement factor (complex field reconstruction by interferometry).
    \item Numerical simulation of a laser signal (amplitude and phase) and its nonlinear propagation in an optical fiber.
    \item Laser wavelength locking (DFB and free space) using the PDH and Tilt-locking techniques on ULE cavities, application to new OEO architectures.
  \end{itemize}
}

\divider

\cvevent{2nd year Master internship}{SensUp (by Lumibird)}{february -- july 2020}{Lannion, France}

\textbf{Caraterization and optimisation of a heterodyne fiber LiDAR}\smallskip

Supervisor : \textbf{Gildas Gueguen} (ggueguen@sensup-tech.com)
\smallskip
\small{
  \begin{itemize}
    \item Caracterization of a fiber LiDAR system (fibers, amplifiers, laser, photodiodes).
    \item Numerical simulation (Matlab) of atmospheric optical propagation (Gaussian beams) and signal analysis.
    \item Softwave development, signal sampling and analysis (Qt/C++) for real-time wind speed (Doppler effect, heterodyne detection) cartography (1D).
  \end{itemize}
}

\divider

\cvevent{1st year Master internship}{OPTIMAG (Université de Bretagne Occidentale)}{april -- june 2019}{Brest, France}

\textbf{Ultra-fast measurement of optical activity by swept-wavelength polarimetry}\smallskip



Supervisor : \textbf{Matthieu Dubreuil} (matthieu.dubreuil@univ-brest.fr)
\smallskip
\small{
  \begin{itemize}
    \item Modelization of a polarimetric setup (Jones formalism and Fourier analysis) which spectrally encodes the optical activity of a sample (chiral) using a swept-wavelength laser source.
    \item Numerical simulation of the system (Mathematica) : sensibility to alignement errors and to noise sources.
  \end{itemize}
}




\cvsection{Education}

\cvevent{Master of Science in Photonics}{Université de Bretagne Occidentale, Université de Rennes}{2019 -- 2020}{Brest--Rennes}
\hfill \textit{With high honors}--\textbf{head of the class}
\vspace{-1em}\small{

  \begin{itemize}
    \item Integrated optics
    \item LASERs \& Telecommunication
    \item Optical propagation  \& Scattering media
    \item Bibliographic project : Supercontinuum generation in micro-structured optical fibers
  \end{itemize}
}

\divider

\cvevent{1st year MSc in fundamental physics and applications}{Université de Bretagne Occidentale}{2018 -- 2019}{Brest}
\hfill\textit{With honours}--\textbf{head of the class}
\vspace{-1em}\small{
  \begin{itemize}
    \item Signal theory
    \item Nonlinear \& anisotropic optics
    \item Statistical physics
    \item Condensed matter
  \end{itemize}
}
\divider

\cvevent{Bachelor of physics}{Université de Bretagne Occidentale}{2015 -- 2018}{Brest}
\hfill \textit{With highest honors}--\textbf{head of the class}
%\small{
%  \begin{itemize}
%    \item Optique ondulatoire
%    \item Mécanique quantique
%    \item Ondes et matière
%    \item Physique expérimentale et numérique
%    \item Projet expérimental de L3 en laboratoire : Caractérisation de biomatériaux par colorimétrie
%  \end{itemize}
%}





%%\divider

%%\cvevent{Agent d'accueil}{CROUS}{septembre 2018 -- juin 2019}{Brest}
%%\textbf{Centre Régional des Œuvres Universitaires et Scolaires}
%%\small{
%%\begin{itemize}
%%\item Accueil
%%\item Distribution du courrier
%%\item Constitution de dossiers administratifs
%%\item Paiements bancaires (CB, chèques)
%%\end{itemize}
%%}

\medskip


\newpage

\begin{fullwidth}

  \cvsection{Peer reviewed journal publications}



  \begin{itemize}
    \item \normalsize \fullcite{Sinquin2021}
          \medskip
    \item \normalsize \fullcite{Sinquin2023}
  \end{itemize}

  % \nocite{*}

  % \begingroup
  % \renewcommand{\section}[2]{}%
  % %\renewcommand{\chapter}[2]{}% for other classes
  % \printbibliography
  % \endgroup
  \cvsection{Peer reviewed international conferences}
  \cvevent{CLEO\textregistered/Europe-EQEC 2021}{Conference on Lasers and Electro-Optics}{21--25 june 2021}{Visioconference}

  \begin{itemize} \item \fullcite{cleo2021}   \end{itemize}

  \smallskip

  {15 minutes talk / visioconference (COVID)}

  \divider




  \cvevent{CLEO\textregistered/Europe-EQEC 2023}{Conference on Lasers and Electro-Optics}{26--30 june 2023}{Munich -- Germany}

  \begin{itemize} \item \fullcite{cleo2023a}   \end{itemize}

  \smallskip


  {15 minutes talk}

  \divider

  \cvevent{CLEO\textregistered/Europe-EQEC 2023}{Conference on Lasers and Electro-Optics}{26--30 june 2023}{Munich -- Germany}

  \begin{itemize} \item \fullcite{cleo2023b}   \end{itemize}

  \smallskip

  {15 minutes talk}

  \cvsection{National conferences (French)}

  \cvevent{Journée du Club Optique Micro-ondes 2021}{Société Française d'Optique}{4 june 2021}{Visioconference}
  {\quote{Signaux Opto-RF très bas bruit de phase et instabilités dynamiques d’un OEO à modulation directe}}

  \smallskip

  {15 minutes talk / visioconference (COVID)} | \href{https://hal.science/hal-03285993}{hal-03285993}

  \divider


  \cvevent{OPTIQUE Dijon 2021}{Société Française d'Optique}{5--9 july 2021}{Dijon -- France}
  {\quote{Oscillateur Opto-Électronique à modulation directe de faible bruit de phase}}

  \smallskip

  {A poster was presented during this conference} | \href{https://hal.science/hal-03284744}{hal-03284744}

  \divider

  \cvevent{OPTIQUE Nice 2022}{Société Française d'Optique}{4--8 july 2022}{Nice -- France}

  {\quote{Génération de peignes de fréquence et d’impulsions dans un Oscillateur Opto-Électronique à modulation directe}}

  \smallskip

  {15 minutes talk} | \href{https://hal.science/hal-03988116}{hal-03988116}

  \divider



  \cvevent{Journée du Club Optique Micro-ondes 2022}{Société Française d'Optique}{13 june 2022}{Besançon -- France}
  {\quote{Oscillateur optoélectronique (OEO) générant des peignes de fréquences et des trains d’impulsions optiques}}
  \medskip

  {A poster was presented during this conference} | \href{https://hal.science/hal-03986413}{hal-03986413}

  \divider

  \cvevent{Journée du Club Optique Micro-ondes 2023}{Société Française d'Optique}{19 june 2023}{Visioconference}
  {\quote{Direct-Modulation OEO for Optical Pulses and Frequency combs generation}}

  \medskip

  {A poster was presented during this online conference} | \href{https://hal.science/hal-04133619}{hal-04133619}

  \newpage

  \cvsection{Other scientific communications}

  \cvevent{Antennes et circuits: des micro-ondes aux ondes millimétriques et THz}{GDR Ondes 2021}{18 march 2021}{Visioconference}

  {\quote{Low phase noise direct-modulation Optoelectronic Oscillator}}

  \smallskip

  {15 minutes talk / visioconference (COVID)} | {\color{red}\faYoutubePlay}  \href{https://www.youtube.com/watch?v=NiCSm7F7ba8}{https://www.youtube.com/watch?v=NiCSm7F7ba8}






  \medskip
  \cvsection{Teaching experience}

  \cvevent{Université de Rennes 1}{UFR SPM}{2020--2022}{Rennes -- France}

  \begin{itemize}
    \item \textbf{Practical Work} - Geometrical Optics \textbf{(1st year Physics Bachelor)} -- 16h
    \item \textbf{Tutorial class} - Electromagnetism \textbf{(2nd year Physics Bachelor)} -- 18h
    \item \textbf{Tutorial class} - Signals and systems theory for the physician \textbf{(3rd year Physics Bachelor)} -- 20h
    \item \textbf{Practical Work} - LASER \textbf{(1st year Physics Master)} -- 4h
  \end{itemize}

  \medskip
  \cvsection{Supervising}

  \cvevent{1st year Master internship}{1 month and a half}{2021}{Rennes -- France}
  \vspace{-0.5em}\textbf{Laser semiconducteur stabilisé sur cavité Fabry-Perot : applications à une
    nouvelle architecture d'OEO}

  \divider


  \cvevent{2nd year Master project}{3 month}{2022}{Rennes -- France}
  \vspace{-0.5em}\textbf{Oscillateur Optoélectronique avec source laser stabilisée par Tilt-Locking}

  \divider

  \cvevent{1st year Master intership}{1 month and a half}{2022}{Rennes -- France}
  \vspace{-0.5em}\textbf{Laser semiconducteur stabilisé sur
    cavité ULE compacte: application à
    de nouvelles architectures d’OEO}

  \divider


  \cvevent{3rd year Bachelor internship}{1 month and a half}{2023}{Rennes -- France}
  \vspace{-0.5em}\textbf{Oscillateur Optoélectronique à modulation directe}




\end{fullwidth}
%% If the NEXT page doesn't start with a \cvsection but you'd
%% still like to add a sidebar, then use this command on THIS
%% page to add it. The optional argument lets you pull up the
%% sidebar a bit so that it looks aligned with the top of the
%% main column.
% \addnextpagesidebar[-1ex]{page3sidebar}


\end{document}
